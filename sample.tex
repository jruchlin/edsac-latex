\documentclass[11pt]{article}
%\usepackage{times}  % for testing
\usepackage{edsac}
\usepackage[normalem]{ulem}

\addtolength{\textheight}{\headheight}
\addtolength{\textheight}{\headsep}
\addtolength{\textheight}{2\topmargin}
\setlength{\headheight}{0pt}
\setlength{\headsep}{0pt}
\setlength{\topmargin}{0pt}

\title{A \LaTeX\ Package for Typesetting EDSAC Programs}
\author{Lee Wittenberg\\\texttt{leew@alumni.stanford.edu}}

\renewcommand{\thefootnote}{\fnsymbol{footnote}}

\newcommand{\sect}[2]{\par\vspace{\baselineskip}#1\quad\uline{#2}}
\newcommand{\hang}{\leftskip=1em\parindent=-1em}
\newenvironment{tape}{%
    \par\begin{tabular}{@{}p{.25\textwidth}%
        >{\hang}p{.565\textwidth}}
}{%
    \end{tabular}\par
}

\begin{document}
\maketitle

\noindent
This document describes and demonstrates the use of \texttt{edsac.sty}, a
(currently experimental) \LaTeX\ package for typesetting EDSAC
programs in a manner similar to that in \emph{The Preparation of
Programs for an Electronic Digital Computer}\footnote{Hereinafter
referred to as WWG.} by Wilkes, Wheeler and
Gill (1951).  The package is loaded with the usual
\begin{quote}
    \verb"\usepackage{edsac}"
\end{quote}
and it provides the
\texttt{edsac} environment and a number of supporting commands.
Package options are described below (Section~\ref{options}).

\section{The \texttt{edsac} Environment}

EDSAC programs are typically typeset with a single order (or control
combination) per line in 3 columns, separated by vertical rules:
the location of the order, the order itself, and an optional comment.
For example, a trivial program loaded at location~56 that does
absolutely nothing:
\begin{edsac}{56}
\nonum  | \T 56 \K | load from location 56  \\
        | \X    \F | do nothing             \\
        | \Z    \F | stop
\end{edsac}
would be specified in \LaTeX\ as:
\begin{quote}\small
\begin{verbatim}
\begin{edsac}{56}
\nonum  | \T 56 \K | load from location 56  \\
        | \X    \F | do nothing             \\
        | \Z    \F | stop
\end{edsac}
\end{verbatim}
\end{quote}
which resembles a typical \texttt{tabular} environment, except that a
vertical bar (`\verb"|"') is used instead of an ampersand (`\verb"&"')
to separate columns.  In fact, a \texttt{tabular} environment is used
to implement the \texttt{edsac} environment.

As you can see, each code letter is specified by the corresponding
single-character
macro (non-alphabetics are specified by macros based on the substitute
characters used
in Martin Campbell-Kelly's simulator: `\verb"\@"' for `$\theta$',
`\verb"\!"' for `$\phi$', `\verb"\&"' for `$\Delta$',
and `\verb"\#"' for `$\pi$').  The perforator characters for ``erase''
and ``blank tape'' are specified by the macros `\verb"\*"' and
`\verb"\."', respectively.  The \verb"\nonum" command is used for
control combinations (which are not placed in the store), and the
environment's single argument specifies the
location of the starting order.

\section{A Second Example}
Campbell-Kelly's ``Hello, World'' program provides a slightly more
complex example:
\begin{edsac}{0}
\nonum       |   \T 64 \K | Load from location 64   \\
\nonum       |   \G    \K | Set $\theta$ parameter  \\
\from{Start} |   \Z    \F | Stop                    \\
             |   \O  5 \@ | Letter shift            \\
             |   \O  6 \@ | Print `H'               \\
             |   \O  7 \@ | Print `I'               \\
             |   \Z    \F | Stop                    \\
             |\| \*    \F | Letters                 \\
             |\| \H    \F | `H'                     \\
             |\| \I    \F | `I'                     \\
\nonum       |   \E    \Z | \longnote{2}{10em}
                              {Enter at location $0\theta$} \\
\nonum       |   \P    \F | 
\end{edsac}
and introduces the \verb"\from" command to indicate the target of a
transfer order and the `\verb"\|"' order column prefix to indicate
constant values.
\begin{quote}\small
\begin{verbatim}
\begin{edsac}{0}
\nonum       |   \T 64 \K | Load from location 64   \\
\nonum       |   \G    \K | Set $\theta$ parameter  \\
\from{Start} |   \Z    \F | Stop                    \\
             |   \O  5 \@ | Letter shift            \\
             |   \O  6 \@ | Print `H'               \\
             |   \O  7 \@ | Print `I'               \\
             |   \Z    \F | Stop                    \\
             |\| \*    \F | Letters                 \\
             |\| \H    \F | `H'                     \\
             |\| \I    \F | `I'                     \\
\nonum       |   \E    \Z | \longnote{2}{10em}
                              {Enter at location $0\theta$} \\
\nonum       |   \P    \F | 
\end{edsac}
\end{verbatim}
\end{quote}

The environment also provides the \verb"\multifrom" command for when an
order is targeted by multiple \verb"E"/\verb"G" orders, as in
subroutine~C7 (WWG, pp.\ 118--20).  The order numbers in the
\verb"\multifrom" argument are separated by \verb"\\"'s.  For example,
line~34 of~C7 would be typeset with ``\verb"\multifrom{30\\31}".''

This example also introduces the \verb"\longnote" command (which can
be used outside of an \texttt{edsac} environment).  It
takes 3 arguments:  (1) the number of rows the note extends over,
(2) the width to allow for the note, and (3) the note itself.  For
a note that must extend onto a following page, an optional argument,
\verb"[t"$n$\verb"]" or \verb"[b"$n$\verb"]", may specify the number
($n<1^{\mbox{st}}$ arg)
of commented lines on the top or bottom, respectively,
of the current page.

\section{More Features}
The \texttt{edsac} environment also provides the order column
prefixes `\verb"\_"' and `\verb"\("'.  The former underlines the order
it introduces, and is used to indicate a transfer of control.  The
latter parenthesizes the order, and is used to indicate that an order
will be changed during execution.  Examples of both are found in the
code for Section~\ref{full-demo}, as are examples of the badly-named
(and not-particularly-well-implemented)
\verb"\eindent" construct, used to keep the columns aligned in consecutive
\texttt{edsac} environments.

The `\verb"\#"' construct is typeset appropriately whether used as a
pseudo-order or to modify a code letter.  For example, the code
\begin{quote}\small
\begin{verbatim}
    | \#     \F | figure shift  \\
    | \A 26\#\@ | add long word at 26
\end{verbatim}
\end{quote}
would be typeset as
\bgroup\setlength{\erulewidth}{0pt}
\begin{edsac}{0}
\nonum  | \#     \F | figure shift  \\
\nonum  | \A 26\#\@ | add long word at 26
\end{edsac}
\egroup

Sometimes, as in the previous example and
the WWG explanation (p.~22) of the ``Wheeler jump,'' the
vertical rules are not wanted, and the starting address is a generic
$m$ or~$n$.
% \enorules is obsolete & will eventually be removed
%The \verb"\enorules" command specifies that the following
%\texttt{edsac} environment will be typeset without vertical rules.
The rigid length \verb"\erulewidth" controls the width of the vertical
rules, and can even be set to \verb"0pt", if desired.
If the \verb"\elineprefix" command is defined at entry, the
\texttt{edsac} environment generates line numbers with the specified
prefix added to all non-zero line numbers.  These commands are
automatically undefined at the end of the environment so that
any program listings that follow will be typeset in the default format.
\bgroup\setlength{\erulewidth}{0pt}
\newcommand{\startloc}{m}
\newcommand{\elineprefix}{\ensuremath{\startloc}}
\begin{edsac}{0}
\multicolumn{1}{c}{Number of}           \\
\multicolumn{1}{c}{storage}             \\
\multicolumn{1}{c}{\uline{location}}
    | \multicolumn{1}{c}{\uline{Order}} \\
        | \A $m\enspace$ \F             \\
        | \G $n\enspace$ \F     \\\nonum\\
\multicolumn{3}{l}{The orders in the subroutine
                   are as follows:}
\\\nonum\global\def\startloc{n}\eresetline\\
\nonum  |   \G       \K     \\
        |   \A    3  \F     \\
        |   \T $p+2$ \@     \\
        |       | \longnote{3}{.4\textwidth}
                    {\hang subroutine operational
                     orders\\($p$ in number)}\\
\multicolumn{1}{c}{\ensuremath{\ldots}}
    \global\def\startloc{n+p}\eresetline[1]\\
        |                   \\
        |\( \Z       \F
\end{edsac}
\egroup

In the corresponding \LaTeX\ code, which follows, note the use of the 
standard \texttt{tabular} \verb"\multicolumn" command, which works as
usual, to interject non-program material into the listing.  Note also
the use of \verb"\eresetline" to adjust line numbers and the plain
\TeX\ command \verb"\global\def" to redefine \verb"\startloc" within
the environment (because \LaTeX's \verb"\renewcommand" only works
locally, within the current column).
\begin{quote}\small
\begin{verbatim}
\setlength{\erulewidth}{0pt}
\newcommand{\startloc}{m}
\newcommand{\elineprefix}{\ensuremath{\startloc}}
\begin{edsac}{0}
\multicolumn{1}{c}{Number of}           \\
\multicolumn{1}{c}{storage}             \\
\multicolumn{1}{c}{\uline{location}}
    | \multicolumn{1}{c}{\uline{Order}} \\
        | \A $m\enspace$ \F             \\
        | \G $n\enspace$ \F     \\\nonum\\
\multicolumn{3}{l}{The orders in the subroutine
                   are as follows:}
\\\nonum\global\def\startloc{n}\eresetline\\
\nonum  |   \G       \K     \\
        |   \A    3  \F     \\
        |   \T $p+2$ \@     \\
        |       | \longnote{3}{.4\textwidth}
                    {\hang subroutine operational
                     orders\\($p$ in number)}\\
\multicolumn{1}{c}{\ensuremath{\ldots}}
    \global\def\startloc{n+p}\eresetline[1]\\
        |                   \\
        |\( \Z       \F
\end{edsac}
\end{verbatim}
\end{quote}

Finally, similar to \verb"\elineprefix", if
\verb"\elinesuffix" is defined prior to an \texttt{edsac} environment
it adds the
specified code at the end of each line number.  This is rarely needed,
but is useful in situations, such as in library subroutine A3, where
integers are placed directly in the code:
\newcommand{\elinesuffix}{\makebox[0pt][l]{\ensuremath{\pi\theta}}}
% avoid conflict with | in edsac environment:
\newcommand{\mc}{\multicolumn{1}{l|}{Part I:}}
\newcommand{\num}[1]
    {\multicolumn{2}{p{7.5em}}{\hfill#1}\skipword}
\begin{edsac}{60}
\mc
    | \multicolumn{1}{c}{\fbox{\parbox[t]{4em}{\centering R9\\modified}}}
    | \parbox[t]{.5\textwidth}{\small\hang order 3 in R2 is altered to
                                     S~40~D.  Thus the integers are
                                     placed in the store negatively
                                     and become $-10^{-n}, n=0\ldots7$} \\
\nonum  | \multicolumn{2}{l}{T $60\pi\theta$ K} \\
        | \num{17 179 869 184 F}                \\
        | \num{ 1 717 986 918 F}                \\
        | \num{   171 798 692 F}                \\
        | \num{    17 179 869 F}                \\
        | \num{     1 717 987 F}                \\
        | \num{       171 799 F}                \\
        | \num{        17 180 F}                \\
        | \num{         1 718 $\pi$}
\end{edsac}

In addition to the use of \verb"\elineprefix", the \LaTeX\ code has
a number of interesting features.  It makes great use of the
\verb"\multicolumn" command (the \verb"\mc" definition is necessary because
a `\verb"|"' in a preamble option causes problems when used within
an \texttt{edsac} environment).  The \verb"\skipword" command (in
\verb"\num")
increments the line number counter for each long word value.
\begin{quote}\small
\begin{verbatim}
\newcommand{\elinesuffix}
    {\makebox[0pt][l]{\ensuremath{\pi\theta}}}
% avoid conflict with | in edsac environment:
\newcommand{\mc}{\multicolumn{1}{l|}{Part I:}}
\newcommand{\num}[1]
    {\multicolumn{2}{p{7.5em}}{\hfill#1}\skipword}
\begin{edsac}{60}
\mc
    | \multicolumn{1}{c}
        {\fbox{\parbox[t]{4em}{\centering R9\\modified}}}
    | \parbox[t]{.5\textwidth}
        {\small\hang order 3 in R2 is altered to
         S~40~D.  Thus the integers are
         placed in the store negatively
         and become $-10^{-n}, n=0\ldots7$}     \\
\nonum  | \multicolumn{2}{l}{T $60\pi\theta$ K} \\
        | \num{17 179 869 184 F}                \\
        | \num{ 1 717 986 918 F}                \\
        | \num{   171 798 692 F}                \\
        | \num{    17 179 869 F}                \\
        | \num{     1 717 987 F}                \\
        | \num{       171 799 F}                \\
        | \num{        17 180 F}                \\
        | \num{         1 718 $\pi$}
\end{edsac}
\end{verbatim}
\end{quote}

\section{Package Options}\label{options}
The package provides two options:  \verb"wwg" and \verb"rulecolor".
The former uses square brackets (\`a la WWG) instead of curly braces
for long comments, the latter specifies a color for rendering the
rules in code listings.  Options may be specified when the package is
loaded (e.g., \verb"\usepackage[wwg]{edsac}") or using the
\verb"\edsacoptions" command.  The following section is rendered
using the options \verb"wwg" and \verb"rulecolor=red".

\section{A Fuller Demonstration}\label{full-demo}
\edsacoptions{wwg,rulecolor=red}
For a fuller demonstration,
Section 7-3 from WWG (pp.~51--53) does the job nicely, and uses most
of the techniques described above (see the \LaTeX\ source for details):

\pagebreak
\sect{\hspace{-\parindent}7-3}{Alternative method for Example 2.}

The example given in Section 7-2 will now be repeated in a revised
form making use of assembly subroutine M1.  The components of the
program are
\begin{quote}\leftskip=.25\textwidth
    H sequence,\\
    Master routine,\\
    Auxiliary subroutine,\\
    Library subroutines R9, Q2, P1, and D6.
\end{quote}
R9, however, is not dealt with by M1 but is automatically placed in
its usual position (56--70).  The H sequence consists of a number of
pseudo-orders which, in Section 7-2 were included at the end of the
master routine.

Storage space is allocated as follows:
\begin{quote}
    \begin{tabular}{c<{\quad}>{\quad}l<{\hspace{-2in}}}
        56--70  &   R9                                      \\
        71--75  &   unused                                  \\
          76    &   reference order for master routine  
                        \qquad\longnote{5}{6em}{%
                          See\\Section\\4-62\hspace{-6em}}  \\
          77    &   \phantom{refernc}do.\phantom{rder for~}
                        \thinspace{}auxiliary               \\
          78    &   \phantom{refernc}do.\phantom{rder for~}
                        \thinspace{}Q2                      \\
          79    &   \phantom{refernc}do.\phantom{rder for~}
                        \thinspace{}P1                      \\
          80    &   \phantom{refernc}do.\phantom{rder for~}
                        \thinspace{}D6                      \\
        82--97  &   M1                                      \\
        98--\phantom{99}
                &   H sequence
    \end{tabular}
\end{quote}

\sect{7-31}{Make up of tape.}\\[2pt]\mbox{}

\begin{tape}
\fbox{R9}        &  R9 begins with P K T 56 K, so that it is placed in
                    locations 56--70.   \\
\\
space
\end{tape}
\begin{tape}\\
P K T 82 K          & First order of M1 goes into 82.       \\
\fbox{M1}\\
P 76 F              & Reference order of master\qquad 
                        \makebox[0pt][l]{\longnote{4}{5em}%
                            {parameters\\used by M1}}       \\
                    & \quad routine goes into 76.           \\
T 98 K              & First order of H sequence             \\
                    & \quad goes into 98.
\end{tape}
\begin{tape}
\\space\\\\
P Z G K \\
E 82 K T F          & Calls in M1, which places P 98 F in 45. \\
\\
\fbox{H sequence}   & Placed with first order in 98.
\end{tape}

\begin{tape}
\\space\\\\
P Z G K \\
T 92 K T $\phi$     & Sets M1 ready to deal with master routine.       \\
E 82 K I F          & Calls in M1, which places reference order in 76. \\
\\
\fbox{Master routine}
\end{tape}
\begin{tape}
\\space\\\\
P Z G K \\
E 82 K P F          & Calls in M1, which places reference order in 77. \\
\\
\fbox{Auxiliary}
\end{tape}
\enlargethispage{.2\baselineskip}
\begin{tape}
\\space\\\\
P Z G K \\
E 82 K P F          & Calls in M1, which places reference order in 78. \\
\\
T 45 F              & \longnote{3}{.45\textwidth}%
                        {Plants parameters required by Q2.} \\
P 72 D \\
P ~1 $\,\phi$ \\
\\
\fbox{Q2}
\end{tape}
\begin{tape}
\\space\\\\
P Z G K \\
E 82 K P F          & Calls in M1, which places reference order in 79. \\
\\
\fbox{P1}
\end{tape}
\begin{tape}
P Z G K \\
E 82 K P F          & Calls in M1, which places reference order in 80. \\
\\
\fbox{D6}
\end{tape}
\begin{tape}
E 25 K              & \longnote{2}{.4\textwidth}%
                        {Sends control to the first order of the
                         master routine.}   \\
E $\phi$ P F
\end{tape}

\sect{7-32}{H sequence.}
\newcommand{\eindent}{\rlap{H}{\phantom{$\mbox{Start}\to{}$}}}
\begin{edsac}{0}
    |\| \&      \F | line feed          \\
    |\| \@      \F | carriage return    \\
    |\| \K 2048 \F | $15\cdot2^{-4}$    \\
    |\| \R      \F | $4\cdot2^{-4}$     \\
    |\| \E      \F | $3\cdot2^{-4}$     \\
    |\| \I      \F | $8\cdot2^{-4}$     \\
    |\| \#      \F | figure shift
\end{edsac}

\sect{7-33}{Master routine.}
\begin{edsac}{0}
\nonum  |   \G     \K |                                     \\
\from{Start}
        |   \O   6 \H | figures                             \\
        |   \O   1 \H | carriage return                     \\
        |   \O     \H | line feed                           \\
        |   \T  72 \D | \longnote{3}{15em}%
                        {\hang Sets limits of integration: 0 to
                         72D,\\1/2 to 74 D.\hspace{-15em}}  \\
        |   \A   5 \H |                                     \\
        |   \T  74 \D |                                     \\
        |   \A   6 \@ | \longnote{2}{6em}{calls in Q2}      \\
        |\_ \G   2 \! |                                     \\
\from{Q2}
        |   \A   8 \@ | \longnote{2}{6em}{calls in P1}      \\
        |\_ \G   3 \! |                                     \\
        |\| \P  10 \F | parameter for P1                    \\
\from{P1}
        |   \O   6 \H | extra output order to print last figure \\
        |   \Z     \F |
\end{edsac}

\sect{7-34}{Auxiliary subroutine}
\newcommand{\eindent}{\enspace\phantom{$\mbox{Start}\to{}$}}
\begin{edsac}{0}
\nonum  |  \G    \K |                               \\
        |  \A  3 \F |                               \\
        |  \T 15 \@ | plant link                    \\
        |  \H    \D |                               \\
        |  \V    \D | $x^2$ to accumulator          \\
        |  \A  3 \H | $4\cdot2^{-4}+x^2$            \\
        |  \Y    \F |                               \\
        |  \T    \D | $0.25+x^2$ to 0D              \\
        |  \H  2 \H | $15\cdot2^{-4}$               \\
        |  \V    \D |                               \\
        |  \Y    \F |                               \\
        |  \T  4 \D | $15/16\cdot(0.25+x^2)$ to 4D  \\
        |  \A  4 \H |                               \\
        |  \T    \D | $3/16$ to 0D                  \\
        |  \A 13 \@ | \longnote{2}{.5\textwidth}%
                        {\hang{}Calls in D6, which places
                         the integrand in~0D.}     \\
        |\_\G  4 \! |                               \\
\from{D6}
      |\_\(\Z    \F | link
\end{edsac}

\section*{Known Issues}
\begin{itemize}
    \item
    The \verb"\eindent" command is very badly named (and its implementation is
    somewhat problematic).  This is also true for a number of the
    other commands.  Suggestions for improvement (both in naming and
    implementation) would be greatly appreciated.
    \item
    It is technically possible to have the characters `P', `Q', etc.\
    declared ``active'' and use them as-is (i.e., without a backslash)
    in the order part of a line.  However, my \TeX{pertise} doesn't
    extent that far (and I'm not completely sure it's a Good Idea).
    Any help in this area would be welcomed.
    \item
    The \texttt{edsac} environment argument really should be
    optional, with zero as the default.  However, \LaTeX\ won't
    recognize `\verb"|"' as a column separator when it is looking for
    the `\verb"["' that would introduce the optional argument value.
    I couldn't figure
    out a clean way to deal with this.  A non-optional argument seemed
    the simplest solution.
    \item
    WWG listings ``stretch'' the arrow that would be used in our
    \verb"\from" command.  Can anyone suggest how to do this in
    \LaTeX?
    \item
    The \texttt{edsac} environment is based on \texttt{tabular}, and
    therefore, cannot extend past the end of the page.  It would be
    nice if \texttt{supertabular} or \texttt{longtable} could be used
    instead.  However,  I wasn't able to accomplish this.
    \item
    All dimensions used in the package are based on the
    characteristics of a document's primary font. Therefore the
    defined commands \emph{should} work correctly regardless of
    typeface and type size.  However, it has only been minimally
    tested---only with Computer Modern and Times Roman at 10, 11 and
    12-point sizes.
\end{itemize}

Any bug reports (and fixes) or suggestions for improvement
should be addressed to the author at \texttt{leew@alumni.stanford.edu}.
Even if you find no bugs, please let me know if you find this package
useful (so I can let you know about any updates and modifications).

\end{document}
